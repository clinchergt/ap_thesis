% Chapter 2

\chapter{Trasfondo del aprendizaje profundo} % Main chapter title

\label{Chapter2} % For referencing the chapter elsewhere, use \ref{Chapter1} 

En este capítulo se planteará y se explicará la base teórica del campo del aprendizaje profundo utilizados en este trabajo. Los conceptos descritos a continuación son el fundamento del campo de aprendizaje profundo


\section{Aprendizaje supervisado}

En el área de aprendizaje de máquina (profundo) existen dos tipos básicos de problemas. El tipo se determina dependiendo de el tipo de datos a utilizar para el entrenamiento del modelo.

Cuando los datos --- ya sea información tabulada, imágenes, textos --- no tienen ninguna categoría o ningún valor a predecir y lo que se desea es obtener información no específica, es decir sin tener alguna referencia, se trata de un aprendizaje no supervisado.

Cuando los datos, por otro lado, tengan una clasificación, llamada etiqueta, asignada, la cual a futuro es el resultado a predecir, los métodos a utilizar son los del aprendizaje supervisado. Debido a que el problema a abordar en este trabajo es un problema de clasificación, el resto del fundamento teórico será basado en el contexto del aprendizaje supervisado.

El aprendizaje supervisado puede entonces ser descrito como una función $f : X \to Y$ donde $X$ representa los datos con los que se alimenta la función, y $Y$ el resultado a asignado o a predecir.

\section{Redes neuronales}

Las redes neuronales son el modelo base para el aprendizaje profundo.

\subsection{Redes neuronales estándar}

\subsection{Redes neuronales recurrentes}