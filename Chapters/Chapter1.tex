% Chapter 1

\chapter{Introducción} % Main chapter title

\label{Chapter1} % For referencing the chapter elsewhere, use \ref{Chapter1} 

%----------------------------------------------------------------------------------------

% Define some commands to keep the formatting separated from the content 
\newcommand{\keyword}[1]{\textbf{#1}}
\newcommand{\tabhead}[1]{\textbf{#1}}
\newcommand{\code}[1]{\texttt{#1}}
\newcommand{\file}[1]{\texttt{\bfseries#1}}
\newcommand{\option}[1]{\texttt{\itshape#1}}

%----------------------------------------------------------------------------------------

\section{Tema principal}

\ttitle.


%----------------------------------------------------------------------------------------

\section{Objetivo General}

Aplicar conceptos del estado del arte a un problema que ha recibido poca atención en el área de procesamiento de lenguaje natural (NLP, por sus siglas en inglés). Esto con el propósito de demostrar la capacidad de las redes neuronales recurrentes (RNN) y \emph{transfer learning} en la tarea específica.

\subsection{Objetivos específicos}

\begin{itemize}
\item Ser pionero en la aplicación de \emph{transfer learning} en la tarea específica de perfilamiento de autores en el área de procesamiento de lenguaje natural.
\item Exponer las ventajas y desventajas de utilizar esta tecnología en esta subtarea.
\item Exponer como caso de uso la utilidad de esta aplicación en distintas áreas, como en el mercadeo.
\item Contribuir a la academia en este campo con los resultados obtenidos
\end{itemize}

%----------------------------------------------------------------------------------------

\section{Introducción}

El campo de \emph{Machine Learning} (aprendizaje de máquinas) y \emph{Deep Learning} (aprendizaje profundo), no son para nada nuevos, pero han sido una gran sensación en los últimos años, cada vez ganando más popularidad. Los primeros conceptos desarrollados en el campo tienen décadas. La primer red neuronal --- la base del modelo a utilizar en este trabajo y la base del aprendizaje profundo --- fue planteada en los años 90. ¿Qué, entonces, ha cambiado en los últimos años que ha generado una explosión en el campo? La disponibilidad de los datos. Con el surgimiento y la penetración cada vez más alta del internet se han abierto las puertas a una cantidad de datos nunca antes vista.

La revolución de los datos ha llevado a la comunidad científica y a las industrias a recolectar grandes cantidades de datos e información y organizarlos de forma que se puedan interpretan y se pueda obtener valor de la misma.


