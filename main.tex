%%%%%%%%%%%%%%%%%%%%%%%%%%%%%%%%%%%%%%%%%
% Masters/Doctoral Thesis 
% LaTeX Template
% Version 2.5 (27/8/17)
%
% This template was downloaded from:
% http://www.LaTeXTemplates.com
%
% Version 2.x major modifications by:
% Vel (vel@latextemplates.com)
%
% This template is based on a template by:
% Steve Gunn (http://users.ecs.soton.ac.uk/srg/softwaretools/document/templates/)
% Sunil Patel (http://www.sunilpatel.co.uk/thesis-template/)
%
% Template license:
% CC BY-NC-SA 3.0 (http://creativecommons.org/licenses/by-nc-sa/3.0/)
%
%%%%%%%%%%%%%%%%%%%%%%%%%%%%%%%%%%%%%%%%%

%----------------------------------------------------------------------------------------
%	PACKAGES AND OTHER DOCUMENT CONFIGURATIONS
%----------------------------------------------------------------------------------------

\documentclass[
12pt, % The default document font size, options: 10pt, 11pt, 12pt
%oneside, % Two side (alternating margins) for binding by default, uncomment to switch to one side
spanish, % ngerman for German
% Double spacing might be required for Galileo *CHECK* - it is required
doublespacing, % Single line spacing, alternatives: onehalfspacing or doublespacing
%draft, % Uncomment to enable draft mode (no pictures, no links, overfull hboxes indicated)
%nolistspacing, % If the document is onehalfspacing or doublespacing, uncomment this to set spacing in lists to single
liststotoc, % Uncomment to add the list of figures/tables/etc to the table of contents
%toctotoc, % Uncomment to add the main table of contents to the table of contents
%%%%% Required by Galileo Uni formatting
parskip, % Uncomment to add space between paragraphs
%%%%%
%nohyperref, % Uncomment to not load the hyperref package
headsepline, % Uncomment to get a line under the header
%chapterinoneline, % Uncomment to place the chapter title next to the number on one line
%consistentlayout, % Uncomment to change the layout of the declaration, abstract and acknowledgements pages to match the default layout
table,
]{MastersDoctoralThesis} % The class file specifying the document structure

\usepackage[utf8]{inputenc} % Required for inputting international characters
%\usepackage[T1]{fontenc} % Output font encoding for international characters
\usepackage{textcomp}

\usepackage{mathpazo} % Use the Palatino font by default

\usepackage[style=apa]{biblatex} % backend is *biber* set your editor to use that

\addbibresource{thesisbib.bib} % The filename of the bibliography

\usepackage[autostyle=true]{csquotes} % Required to generate language-dependent quotes in the bibliography

\usepackage{xcolor} % Color alternating in tables

\usepackage{comment}

\usepackage{arydshln} % Dashed lines in tables

\setcounter{secnumdepth}{3} % Para que las subsubsections tengan número

\makeatletter
\newcommand{\tblshort}{\let\blx@imc@ifciteseen\@firstoftwo}
\makeatother

%----------------------------------------------------------------------------------------
%	MARGIN SETTINGS
%----------------------------------------------------------------------------------------

\geometry{
	paper=letterpaper, % Change to letterpaper for US letter
	inner=2.5cm, % Inner margin
	outer=3.8cm, % Outer margin
	bindingoffset=.5cm, % Binding offset
	top=1.5cm, % Top margin
	bottom=1.5cm, % Bottom margin
	%showframe, % Uncomment to show how the type block is set on the page
}

%----------------------------------------------------------------------------------------
%	THESIS INFORMATION
%----------------------------------------------------------------------------------------

\thesistitle{Aplicación de transferencia de aprendizaje en redes neurales al tema de perfilación de autores de textos anónimos} % Your thesis title, this is used in the title and abstract, print it elsewhere with \ttitle
%\supervisor{Ing. Alí \textsc{Lemus}} % Your supervisor's name, this is used in the title page, print it elsewhere with \supname
\supervisor{Ing. Ronald \textsc{López}} % Your supervisor's name, this is used in the title page, print it elsewhere with \supname
\examiner{} % Your examiner's name, this is not currently used anywhere in the template, print it elsewhere with \examname
\degree{Ingeniería de Sistemas, Informática y Ciencias de la Computación} % Your degree name, this is used in the title page and abstract, print it elsewhere with \degreename
\author{José Jacobo \textsc{Del Valle Girón}} % Your name, this is used in the title page and abstract, print it elsewhere with \authorname
\addresses{} % Your address, this is not currently used anywhere in the template, print it elsewhere with \addressname

\subject{Ciencias de la computación} % Your subject area, this is not currently used anywhere in the template, print it elsewhere with \subjectname
\keywords{} % Keywords for your thesis, this is not currently used anywhere in the template, print it elsewhere with \keywordnames
\university{\href{http://galileo.edu}{Universidad Galileo}} % Your university's name and URL, this is used in the title page and abstract, print it elsewhere with \univname
\department{\href{http://www.galileo.edu/fisicc/carrera/ii/}{Ingeniería de sistemas}} % Your department's name and URL, this is used in the title page and abstract, print it elsewhere with \deptname
\group{\href{http://turing.galileo.edu/}{Turing Laboratory}} % Your research group's name and URL, this is used in the title page, print it elsewhere with \groupname
\faculty{\href{http://galileo.edu/fisicc}{FISICC}} % Your faculty's name and URL, this is used in the title page and abstract, print it elsewhere with \facname

\AtBeginDocument{
\hypersetup{pdftitle=\ttitle} % Set the PDF's title to your title
\hypersetup{pdfauthor=\authorname} % Set the PDF's author to your name
\hypersetup{pdfkeywords=\keywordnames} % Set the PDF's keywords to your keywords
%\hypersetup{urlcolor=blue} % URL color
\hypersetup{colorlinks=false} % URL color
}

\begin{document}

\frontmatter % Use roman page numbering style (i, ii, iii, iv...) for the pre-content pages

\pagestyle{plain} % Default to the plain heading style until the thesis style is called for the body content

%----------------------------------------------------------------------------------------
%	TITLE PAGE
%----------------------------------------------------------------------------------------

\begin{titlepage}
\begin{center}

\begingroup\onehalfspacing

%\textsc{\Large Proyecto de examen privado}\\[0.5cm] % Thesis type
{\scshape\LARGE \authorname\par}\vspace{1.5cm} % Author name

{\huge \bfseries \ttitle\par} % Thesis title

\vspace{1.5cm}

\includegraphics[scale=0.5]{logo.png} % University/department logo

\vspace{1.5cm}

{\scshape\Large \univname\par} % University name
{\scshape\Large Facultad de Ingeniería en Sistemas,}
{\scshape\Large Informática y Ciencias de la Computación\par}

{\scshape \Large \bfseries Guatemala, \the\year \par}

\endgroup
 
\vfill
\end{center}
\end{titlepage}

%----------------------------------------------------------------------------------------
%	DECLARATION PAGE
%----------------------------------------------------------------------------------------

%\begin{declaration}
%\addchaptertocentry{\authorshipname} % Add the declaration to the table of contents
%\noindent I, \authorname, declare that this thesis titled, \enquote{\ttitle} and the work presented in it are my own. I confirm that:
%
%\begin{itemize} 
%\item This work was done wholly or mainly while in candidature for a research degree at this University.
%\item Where any part of this thesis has previously been submitted for a degree or any other qualification at this University or any other institution, this has been clearly stated.
%\item Where I have consulted the published work of others, this is always clearly attributed.
%\item Where I have quoted from the work of others, the source is always given. With the exception of such quotations, this thesis is entirely my own work.
%\item I have acknowledged all main sources of help.
%\item Where the thesis is based on work done by myself jointly with others, I have made clear exactly what was done by others and what I have contributed myself.\\
%\end{itemize}
% 
%\noindent Signed:\\
%\rule[0.5em]{25em}{0.5pt} % This prints a line for the signature
% 
%\noindent Date:\\
%\rule[0.5em]{25em}{0.5pt} % This prints a line to write the date
%\end{declaration}
%
%\cleardoublepage

%----------------------------------------------------------------------------------------
%	QUOTATION PAGE
%----------------------------------------------------------------------------------------
%
%\vspace*{0.2\textheight}
%
%\noindent\enquote{\itshape Thanks to my solid academic training, today I can write hundreds of words on virtually any topic without possessing a shred of information, which is how I got a good job in journalism.}\bigbreak
%
%\hfill Dave Barry

%----------------------------------------------------------------------------------------
%	ABSTRACT PAGE
%----------------------------------------------------------------------------------------

\begin{abstract}
\addchaptertocentry{\abstractname} % Add the abstract to the table of contents
En tiempos de la expansión masiva del campo del aprendizaje profundo y considerando que las áreas de aplicación son cada vez más, cada contribución y nueva aplicación al campo no sorprenden. Esto se debe al crecimiento exponencial de datos disponibles en esta época de la internet, lo cual es relevante para el aprendizaje profundo ya que algo que lo caracteriza es la necesidad masiva de datos que se necesitan para poder tener un modelo eficaz. Algunas áreas no cuentan con la cantidad de datos que se necesitan, por lo que en la actualidad se utilizan aún técnicas de \textit{Machine Learning} tradicional para obtener los resultados deseados. Una de estas áreas es el procesamiento de lenguaje natural (NLP por sus siglas en inglés).

Este trabajo presenta la tarea del perfilamiento de autores como una tarea que puede ser abordada con aprendizaje profundo y que también puede aplicarse el concepto de \textit{transfer learning} en el área de NLP de manera exitosa. Este concepto permite poder realizar tareas con menos datos de lo que antes asumido.

Los resultados obtenidos con este proyecto alcanzan niveles del estado del arte de técnicas de aprendizaje profundo y son comparables con resultados obtenidos utilizando técnicas tradicionales y de aprendizaje profundo.
\end{abstract}

%----------------------------------------------------------------------------------------
%	ACKNOWLEDGEMENTS
%----------------------------------------------------------------------------------------

%\begin{acknowledgements}
%\addchaptertocentry{\acknowledgementname} % Add the acknowledgements to the table of contents
%Hay muchas personas a quienes agradecer y aunque tal vez no las logre mencionar a todas aca, principalmente quiero decir que le agradezco a todas las personas que en algún momento me apoyaron durante la carrera o durante la elaboración de este trabajo.
%
%Primero debo agradecerle a mis compañeros en la carrera sin quienes no hubiera llegado a donde estoy. En particular a personas como\ldots
%
%Debo agradecerles también a mis asesores a través de este proceso: Ronald López y Alí Lemus. A Ronald López por haberme apoyado y guiado durante el proceso de la elaboración tesis. Alí por haber apoyado el enfoque académico que siempre quise darle a este trabajo de investigación.
%\end{acknowledgements}

%----------------------------------------------------------------------------------------
%	LIST OF CONTENTS/FIGURES/TABLES PAGES
%----------------------------------------------------------------------------------------
\begingroup\onehalfspacing
\tableofcontents % Prints the main table of contents
\endgroup

\listoffigures % Prints the list of figures

\listoftables % Prints the list of tables

%%----------------------------------------------------------------------------------------
%%	ABBREVIATIONS
%%----------------------------------------------------------------------------------------

%\begin{abbreviations}{ll} % Include a list of abbreviations (a table of two columns)
%
%\textbf{CNN} & \textbf{C}onvolutional \textbf{N}eural \textbf{N}etwork\\
%\textbf{CV} & \textbf{C}omputer \textbf{V}ision\\
%\textbf{DL} & \textbf{D}eep \textbf{L}earning\\
%\textbf{FIFO} & \textbf{F}irst \textbf{I}n \textbf{F}irst \textbf{O}ut\\
%\textbf{GIGO} & \textbf{G}arbage \textbf{I}n \textbf{G}arbage \textbf{O}ut\\
%\textbf{LEFTy} & \textbf{L}anguage \textbf{EF}ficient \textbf{T}ext portra\textbf{y}\\
%\textbf{LSTM} & \textbf{L}ong \textbf{S}hort \textbf{T}erm \textbf{M}emory\\
%\textbf{ML} & \textbf{M}achine \textbf{L}earning\\
%\textbf{NLP} & \textbf{N}atural \textbf{L}anguage \textbf{P}rocessing\\
%\textbf{QRNN} & \textbf{Q}uasi-\textbf{R}ecurrent \textbf{N}eural \textbf{N}etwork\\
%\textbf{RNN} & \textbf{R}ecurrent \textbf{N}eural \textbf{N}etwork\\
%\textbf{ULMFiT} & \textbf{U}niversal \textbf{L}anguage \textbf{M}odel \textbf{Fi}ne-\textbf{T}uning\\
%
%\end{abbreviations}

%%----------------------------------------------------------------------------------------
%%	PHYSICAL CONSTANTS/OTHER DEFINITIONS
%%----------------------------------------------------------------------------------------
%
%\begin{constants}{lr@{${}={}$}l} % The list of physical constants is a three column table
%
%% The \SI{}{} command is provided by the siunitx package, see its documentation for instructions on how to use it
%
%Speed of Light & $c_{0}$ & \SI{2.99792458e8}{\meter\per\second} (exact)\\
%%Constant Name & $Symbol$ & $Constant Value$ with units\\
%
%\end{constants}
%
%%----------------------------------------------------------------------------------------
%%	SYMBOLS
%%----------------------------------------------------------------------------------------
%
%\begin{symbols}{lll} % Include a list of Symbols (a three column table)
%
%$a$ & distance & \si{\meter} \\
%$P$ & power & \si{\watt} (\si{\joule\per\second}) \\
%%Symbol & Name & Unit \\
%
%\addlinespace % Gap to separate the Roman symbols from the Greek
%
%$\omega$ & angular frequency & \si{\radian} \\
%
%\end{symbols}

%----------------------------------------------------------------------------------------
%	DEDICATION
%----------------------------------------------------------------------------------------

%\dedicatory{Dedicado a mi familia, amigos y a la comunidad científica lingüística} 

%----------------------------------------------------------------------------------------
%	THESIS CONTENT - CHAPTERS
%----------------------------------------------------------------------------------------

\mainmatter % Begin numeric (1,2,3...) page numbering

\pagestyle{thesis} % Return the page headers back to the "thesis" style

% Custom indentation for Galileo University
\setlength\parindent{2em}

% Include the chapters of the thesis as separate files from the Chapters folder
% Uncomment the lines as you write the chapters

% Chapter 1

\chapter{Introducción} % Main chapter title

\label{Chapter1} % For referencing the chapter elsewhere, use \ref{Chapter1} 

%----------------------------------------------------------------------------------------

% Define some commands to keep the formatting separated from the content 
\newcommand{\keyword}[1]{\textbf{#1}}
\newcommand{\tabhead}[1]{\textbf{#1}}
\newcommand{\code}[1]{\texttt{#1}}
\newcommand{\file}[1]{\texttt{\bfseries#1}}
\newcommand{\option}[1]{\texttt{\itshape#1}}

%----------------------------------------------------------------------------------------

\section{Tema principal}

\ttitle.


%----------------------------------------------------------------------------------------

\section{Objetivo General}

Aplicar conceptos del estado del arte a un problema que ha recibido poca atención en el área de procesamiento de lenguaje natural (NLP, por sus siglas en inglés). Esto con el propósito de demostrar la capacidad de las redes neuronales recurrentes (RNN) y \emph{transfer learning} en la tarea específica.

\subsection{Objetivos específicos}

\begin{itemize}
\item Ser pionero en la aplicación de \emph{transfer learning} en la tarea específica de perfilamiento de autores en el área de procesamiento de lenguaje natural.
\item Exponer las ventajas y desventajas de utilizar esta tecnología en esta subtarea.
\item Exponer como caso de uso la utilidad de esta aplicación en distintas áreas, como en el mercadeo.
\item Contribuir a la academia en este campo con los resultados obtenidos
\end{itemize}

%----------------------------------------------------------------------------------------

\section{Introducción}

El campo de \emph{Machine Learning} (aprendizaje de máquinas) y \emph{Deep Learning} (aprendizaje profundo), no son para nada nuevos, pero han sido una gran sensación en los últimos años, cada vez ganando más popularidad. Los primeros conceptos desarrollados en el campo tienen décadas. La primer red neuronal --- la base del modelo a utilizar en este trabajo y la base del aprendizaje profundo --- fue planteada en los años 90. ¿Qué, entonces, ha cambiado en los últimos años que ha generado una explosión en el campo? La disponibilidad de los datos. Con el surgimiento y la penetración cada vez más alta del internet se han abierto las puertas a una cantidad de datos nunca antes vista.

La revolución de los datos ha llevado a la comunidad científica y a las industrias a recolectar grandes cantidades de datos e información y organizarlos de forma que se puedan interpretan y se pueda obtener valor de la misma.

Los sistemas de aprendizaje profundo necesitan pasar por una fase de aprendizaje (comunmente llamada fase de entrenamiento) y estos modelos tienen la característica de necesitar cantidades masivas de datos para poder generalizar bien lo aprendido en su fase de entrenamiento. Esto combinado con la explosión de datos ha llevado a que el campo tenga un surgimiento en todas las áreas de software, mostrando mucha promesa en algunos campos en específico como \emph{Computer Vision} (visión de computadoras, CV) y en NLP.

Conforme ha avanzado el campo se ha determinado que para algunas tareas en específico no se cuenta con la cantidad necesaria para poder obtener un modelo competente. Esto ha llevado al desarrollo de una técnica llamada \emph{Transfer Learning} que permite utilizar un modelo pre-entrenado en una tarea más general que la deseada y poder realizar ajustes específicos para una tareá más acotada. Este proceso de ajuste requiere de una cantidad de datos considerablemente menor a entrenar un modelo completamente desde cero.

La técnica de \emph{transfer learning} ha permitido muchos avances en el área de CV específicamente. Este trabajo intenta mostrar la habilidad que se tiene de aplicar este concepto en un área distinta como NLP, específicamente en la tarea de perfilamiento de autores.

\section{Planteamiento del problema}

En el área de NLP han existido técnicas poderosas por décadas que abusan de algunas suposiciones que no siempre son ciertas y se basan en simples métodos estadísticos que asignan valores a características del texto. El campo también ha dependido de ingeniería de características en los datos, lo cual significa que hay esfuerzos activos serios en analizar cada sub-tarea a fondo y determinar cuáles serían los mejores determinantes de un resultado deseado, p.e. las carecterísticas principales que determinan la categoría que se le debe asignar un texto.

En áreas como mercadeo y ciencias forenses hay una necesidad de identificar textos escritos por autores desconocidos, obtener características de los mismos, y poder categorizarlos de forma no manual. Esto con el fin de poder, en el caso de mercadeo, orientar mejor la publicidad de ciertos productos a una audiencia apropriada. En el área forense con el fin de identificar a sospechosos vinculados a algún crimen a partir de un texto anónimo obtenido en el contexto del caso.








% Chapter 2
\chapter{Marco teórico} % Main chapter title

\label{Chapter2} % For referencing the chapter elsewhere, use \ref{Chapter1} 

\section{Trasfondo del aprendizaje profundo} % Main chapter title

En este capítulo se planteará y se explicará la base teórica del campo del aprendizaje profundo utilizados en este trabajo. Los conceptos descritos a continuación son el fundamento del campo de aprendizaje profundo.

El trasfondo que se compartirá en este trabajo es el fundamento teórico de las técnicas utilizadas; los fundamentos geométricos teóricos \parencite{2018leigeometric} del porqué de la funcionalidad básica de los modelos del campo no se abordarán en esta tesis.

\section{Aprendizaje supervisado}

En el área de aprendizaje (profundo) artificial existen dos tipos básicos de problema. Este se determina dependiendo de las características de los datos a utilizar para el entrenamiento del modelo \parencite{schmidhuber2015deep}.

Cuando los datos --- ya sea información tabulada, imágenes, textos --- no tienen ninguna categoría asociada o ningún valor a predecir y lo que se desea es obtener información no específica, es decir sin tener alguna referencia, se trata de un aprendizaje no supervisado.

Si los datos, por otro lado, tienen una clasificación  --- llamada etiqueta --- asignada, la cual a futuro es el resultado a predecir, los métodos a utilizar son los del \gls{aprendizaje supervisado}. Debido a que el problema a abordar en este trabajo es un problema de clasificación, el resto del fundamento teórico será basado en este tipo de contexto.

El aprendizaje supervisado puede entonces ser descrito como una función $f : X \to Y$ donde $X$ representa los datos con los que se alimenta la función, es decir las características que utilizará el modelo para inferir, y $Y$ el resultado asignado o a predecir.

Para ilustrar el proceso de aprendizaje profundo se utilizará un caso individual de la función en donde $y$ es el resultado deseado, $f(x) = \hat{y}$ es la función aplicada a un caso específico y $\hat{y}$ representa el resultado obtenido con la función el cual no necesariamente es el resultado deseado o esperado.

\textbf{El objetivo.} En concreto, buscamos una función $f$ que sea la mejor candidata para poder predecir los resultados deseados. Si definimos una función de costo $L(\hat{y}, y)$ que representa, en un valor escalar, la diferencia cuantitativa entre la evaluación de una función $f$ candidata y el resultado real $y$ podemos concluir que el objetivo es encontrar una $f^*$ que cumpla con:

\[ f^* = \min_{f \in F} \frac{1}{N} \sum_{i = 0}^{N} L(f(x_i), y_i) \]

Donde $N$ es el número de instancias de los datos para entrenar el modelo; $F$ siendo un conjunto de funciones candidatas.

\textbf{Definición de las funciones.} La base de todo modelo de aprendizaje profundo es una red neuronal --- cuyo comportamiento será definido en la siguiente sección así como otros detalles relevantes --- y su comportamiento básico puede ser descrito de la siguiente forma:

\[ f(x_i) = w x_i + b \]

Donde $x_i$ es la instancia de datos $i$ con propósitos de entreno o de predicción. Esto nos dice que $w$ y $b$ serán los parámetros a modificar de una manera sistemática para encontrar la función $f^*$. Con fines de brevedad, la concatenación de $w$ y $b$ serán representados por $\theta$.

Una posible función de pérdida en un problema de categorización para una predicción obtenida toma la forma del error de la entropía cruzada, es decir:

\[ L(\hat{y_i}, y_i) = y_i log\hat{y_i} + (1 - y_i)log(1 - \hat{y_i}) \]

Al tener la pérdida para una predicción se puede expandir esta idea para obtener la pérdida a través de un conjunto de datos, lo cual resultará muy útil cuando se deba entrenar. Para obtener una aproximación de la perdida sobre un conjunto de datos se puede utilizar el promedio sobre las perdidas individuales de los datos evaluados:

\[ L(\hat{y}, y) = - \frac{1}{N} \sum_{i = 1}^{N} [y_i log\hat{y_i} + (1 - y_i)log(1 - \hat{y_i})] \]

\textbf{Optimización de la función de costo.} Con una función a minimizar establecida y una cantidad $N$ de datos sobre los cuales se debe encontrar una función $f$ candidata cada vez mejor, se recurre al método del descenso de gradiente. Este método nos permite, utilizar propiedades básicas de las derivadas de las funciones y poco a poco avanzar hasta llegar a un valor mínimo de la función. Cada nueva función candidata entonces podrá ser derivada de la siguiente forma:

\[ f_i(x_i) = \theta_i x_i \]

Donde
\begin{equation}
\label{eq:sgdupdate}
\theta_{i + 1} = \theta_{i} - \gamma \nabla_{\theta} L(\hat{y_i}, y_i).
\end{equation}

Este proceso de utilizar el descenso de gradiente a través de las instancias nos permite  minimizar el error de la función hasta poder deducir la función que muestra el menor error, lo cual fue descrito como el objetivo principal.

\textbf{La \gls{tasa de aprendizaje}.} La velocidad de convergencia de este proceso dependerá en gran parte de $\gamma$ que representa la tasa de aprendizaje; específicamente es la ponderación que se le da al componente del gradiente cuando se propone la nueva función. Un $\gamma$ muy alto arriesga una divergencia debido a que podría oscilar alrededor de un mínimo sin nunca poder converger en él. Un $\gamma$ muy bajo, por el otro lado, puede resultar en un aprendizaje muy lento, lo cual puede llevar a un resultado no óptimo debido a limitaciones de recursos. Este concepto será importante en capítulos posteriores de este trabajo.

\section{Redes neuronales}

Las redes neuronales son el modelo base para el aprendizaje profundo. Esto debido a que ha sido demostrado que son aproximadores universales \parencite{hornik1989universal} --- es decir que para cualquier función $h$ se tiene la capacidad de encontrar un red neuronal $\hat{H}$ que aproxime a $h$ con cierto grado de precisión --- un error de $\epsilon$ --- dado que se tenga la cantidad de unidades en la red y datos suficientes para entrenar. Formalmente se dice que

\begin{equation}
\label{eq:universaltheorem}
\vert \hat{H}(x) - h(x) \vert < \epsilon
\end{equation}

Con ayuda del concepto de las redes neuronales especificaremos más acerca de la función $f$ que hasta ahora ha permanecido general, únicamente con la restricción de ser derivable. Hay diferentes tipos de redes neuronales en el campo y en este trabajo se abordarán únicamente las redes neuronales estándar y las recurrentes.

\subsection{Redes neuronales estándar}

\begin{figure}
	\includegraphics[scale=.6]{Figures/standardnn.pdf}
	\caption{Una red neuronal estándar con una capa oculta, la cual tiene tres unidades neuronales. La red está completamente conectada, tiene tres nodos de entrada y uno de salida.}
	\label{fig:standardnn}
\end{figure}

\textbf{Inspiración e intuición.} Estas redes neuronales fueron basadas en comportamientos biológicos y reflejan un comportamiento similar a la comunicación de neuronas que se aprecia en la naturaleza. La entrada de datos en una neurona es procesada y alimentará a la siguiente neurona y así sucesivamente hasta haber recorrido toda la red. La redes neuronales no son secuencias directas y lineales de neuronas; las redes están divididas en capas, las cuales pueden contener más de una unidad. En una red completamente conectada, como la que se aprecia en la figura \ref{fig:standardnn}, conecta a todas las unidades de una capa con todas las unidades de la siguiente. Así como las neuronas en la naturaleza, las neuronas en una red neuronal tienen una unidad o célula principal, axones y su conexión es llamada sinapsis.

\textbf{Propagación hacia adelante.} En las redes neuronales esto es el proceso del flujo de la información a través de la red, y las funciones que la componen, hasta obtener un resultado. En el caso de una red neuronal típica, este proceso significa que la salida de una neurona en una capa alimenta parcialmente a las de la capa siguiente. La función a utilizar por cada unidad individual es una regresión lineal simple que puede ser descrita como $f(x) = \mathbf{w} \mathbf{x} + b$, es decir, una matriz de parámetros a multiplicarse con los datos de entrada a la neurona. Adicional a esto se maneja una función de activación para el resultado de esa multiplicación, la cual es no lineal, a cada elemento resultante (p.e. $\tanh$). De esto se concluye que para, por ejemplo, obtener el resultado de una neurona en la cuarta capa se tiene que
\begin{equation}
\label{eq:feedfwdeq}
f(\mathbf{x}) = \mathbf{W_4} \sigma(\mathbf{W_3} \sigma (\mathbf{W_2} \sigma(\mathbf{W_1} \mathbf{x})))
\end{equation}

Donde $\mathbf{W_4}$ es una matriz de parámetros de la cuarta capa, $\mathbf{W_3}$ es una matriz de parámetros de la tercera capa, y así sucesivamente, y $\mathbf{x}$ representa los datos de entrada.

%Figura demostrando fwd y back prop
\begin{figure}
\includegraphics[scale=0.8]{Figures/backprop.pdf}
\caption{En azul se puede ver la dirección del flujo de información durante la propagación hacia adelante. En rojo se aprecia que la dirección se invierte para la propagación hacia atrás y que lleva los gradientes necesarios para propagar el error después del entrenamiento.}
\label{fig:backprop}
\end{figure}


\textbf{Propagación hacia atrás} \parencite{rumelhart1986learning}. Ya que está presente un mecanismo para evaluar el conjunto de funciones que representa cada unidad de la red, debemos tener un mecanismo para optimizar los parámetros que definen cada función. La propagación hacia atrás se encarga de esto utilizando el concepto previamente descrito como el descenso de gradiente. Para realizar la propagación hacia atrás se aplica la regla de la cadena la cual establece que $\frac{\partial f(g(x))}{\partial x} = \frac{\partial g}{\partial x} \frac{\partial f}{\partial g}$. Se debe notar que la función que se quiere derivar toma una forma similar a la que tenemos en la ecuación \ref{eq:feedfwdeq}, con la adición de que $g(x)$ es una función que anida aún más funciones. Para obtener la optimización de los parámetros podemos derivar en dirección hacia atrás propagando la mejora que se propone con el gradiente.

\subsection{Redes neuronales recurrentes}

% Figura con una LSTM y su descripcion


Este tipo de redes tienen la peculiaridad que se alimentan no solamente de los resultados de las funciones de activación de las capas anteriores sino también del resultado de la instancia previamente evaluada; para datos evaluados en en un tiempo $t$ la definición sería $h_t = f_{\theta}(h_{t-1}, x_t)$. Este tipo de estructura de red neuronal es aplicado a conjuntos de datos secuenciales como lo es el procesamiento de textos --- textos cuya representación consiste en una secuencia de palabras representadas de forma vectorial --- y procesamiento de secuencias de señales.

Se debe aclarar para esta estructura de red neuronal también existen distintas extensiones. El subtipo relevante para este trabajo de investigación es la red \gls{lstm} -- memoria corta a largo plazo. Estas redes tienen un dato en memoria en cada una de las unidades de la red. Los parámetros con los que se decide si se reemplaza lo que está en memoria en esa celda en ese momento son entrenados de la misma manera con el mecanismo de propagación hacia atrás.

Los mecanismos de propagación hacia adelante y hacia atrás permanecen iguales pero se deberán tomar en cuenta adicionalmente las puertas (\textit{gates}) con sus funciones de activación. Estas son definidas como la función sigmoide, la cual es continua y tiene un contradominio de $[0,1]$ lo cual la hace derivable en todos sus puntos.

\section{Generalización de una red neuronal}

Una red neuronal aprende a elegir una función que minimice el costo de evaluar un conjunto de datos. Intuitivamente podemos ver que este proceso lleva a que la red aprenda características de los datos que está utilizando y poco a poco aprenda a predecir categorías basado en este conjunto de datos de manera más precisa. Las palabras claves son \textit{este conjunto de datos}, es decir se habla de que la red aprende sobre un conjunto limitado de datos y fuera de él no hay garantía de que sea competente. Para que el modelo pueda generalizar lo aprendido con estos datos existen distintos mecanismos. En esta sección se explicarán dos de estos conceptos.

\textbf{Cantidad de datos.} La cantidad de datos utilizados para entrenar un modelo influye mucho en su capacidad de generalizar. Mientras más datos se tengan, mayor será la posibilidad de generalizar, esto con la condición que la data sea diversa y representativa del problema real. La explicación de esto se puede ilustrar llevando el concepto a sus extremos y con un ejemplo sencillo: se debe suponer que se quiere aprender a definir el conectivo lógico \textbf{\textit{and}}. Si se provee solamente un ejemplo de cómo funciona este operador, la red no sabrá qué hacer cuando los valores de entrada difieran de ese ejemplo. Por el otro lado si se le alimentan todas las combinaciones posibles, la red deberá ser capaz de aprender todo el contexto del problema.

\textbf{Término de regularización.} Esta técnica es muy esencial cuando se lidia con modelos de aprendizaje profundo. Consisten en agregar un término de regularización a la función de predicción que se está optimizando. La función entonces tendrá la forma siguiente: %hacer referencia a la misma de antes

\[ f^* = \min_{f \in F} \frac{1}{N} \sum_{i} L(f(x_i), y_i) + \sum_{j} \lambda(w_j^2) \]

Esto funciona ya que limita el crecimiento de los parámetros, el cual desenfrenado podría causar que el modelo dependa mucho de una característica de los datos y una dependencia exagerada puede llevar a falta de generalización. En otras palabras, mantiene un balance sobre la ponderación de los parámetros aprendidos. Adicionalmente se incluye el término $\lambda$ el cuál controla qué tanto afecta el término de regularización a la función de pérdida.

\section{Proceso general al aplicar una red neuronal}
\label{sec:nlpprocess}

Un proyecto de aprendizaje profundo con redes neuronales lleva por lo general el mismo conjunto de pasos para poder llegar a un resultado cercano a lo óptimo. Los pasos a seguir son los siguientes:

\begin{itemize}
\item \textbf{Obtención de datos:} Dependiendo del problema a resolver, estos datos podrán tomar distintas formas y los métodos para obtenerlos podrán variar en gran manera. Para que una red neuronal pueda generalizar de forma exitosa lo aprendido durante la fase de entrenamiento es importante tener una muestra representativa del escenario real del problema a resolver y tener una cantidad elevada de datos. Métodos comunes incluyen recolección y etiquetación manual, descarga de \textit{corpora} de internet y \textit{\gls{scraping}} de la web.

\item \textbf{Análisis y preparación de los datos:} Los datos obtenidos en el primer paso pueden llegar a tener características mínimas no deseadas, las cuales pueden añadir ruido a la representación que estos datos dan. Debido a esto es importante tratar los datos de manera que se eliminen datos que sesguen mucho los posibles modelos, datos faltantes, datos con formato inconsistente, e incluso considerar la posibildad de eliminar características completas. En esta fase también se separarán los datos en distintos segmentos que después serán útiles para determinar la eficacia del modelo resultante. Estos segmentos son los datos de \textit{entrenamiento}, \textit{validación} y \textit{prueba}. La proporción de cada uno de estos puede variar y lo recomendado es que la distribución de cada uno de ellos sea la misma. De no ser posible esto, hacer que al menos los segmentos de validación y prueba tengan la misma distribución.

\item \textbf{Diseño de modelo:} En este campo hay una gran variedad de opciones, en especial si no se limitan estas opciones al aprendizaje profundo ya que existen herramientas de distintos tipos para poder modelar un sistema. En el caso del aprendizaje profundo también se deben tomar decisiones importantes con respecto al modelo a utilizar. En concreto se deberá elegir el tipo de red neuronal a utilizar como también su estructura. En esta fase se definirán de forma preliminar detalles como el número de capas a utilizar en la red y el número de unidades por cada una de las capas.

\item \textbf{Optimización de hiper-parámetros:} Esta fase estará muy relacionada a la siguiente ya que se utilizan los mismo mecanismos para determinar los resultados preliminares. Para elegir los mejores hiper-parámetros para un modelo en específico se deberá evaluar los datos variando sus valores y obteniendo resultados preliminares. Esto se deberá hacer utilizando una matriz aleatoria para los distintos hiper-parámetros con el fin de optimizar el tiempo de ejecución de esta fase.

\item \textbf{Entrenamiento del modelo:} En esta fase se optimizarán los parámetros de la red para que esta sea capaz de modelar el problema y poder realizar predicciones. En esta fase se aplicarán de forma iterativa las fases de propagación hacia adelante y hacia atrás hasta llegar a una convergencia. Si se cuenta con tiempo limitado, se deberá elegir un resultado que sea suficientemente satisfactorio y se detendrá el entrenamiento en ese punto.

\item \textbf{Validación de resultados:} Durante esta fase se hará uso de los datos segmentados para validación. Sobre este conjunto se aplicará el modelo en su modalidad de propagación hacia adelante únicamente, con lo que se obtienen predicciones. Estas se pueden comparar con las etiquetas asociadas respectivamente. El fin de evaluar el modelo en datos que no están presentes en el conjunto de entrenamiento es determinar el poder de generalización del modelo.

\end{itemize}

Los pasos de entrenamiento y validación de resultados podrán repetirse las veces que sean necesarias para poder variar los parámetros y evaluar los tipos de errores para mejorar los resultados.

\section{Definiciones varias}

En esta sección se definen conceptos varios que son mencionados y utilizados en este trabajo. Las explicaciones varían en su profundidad dependiendo de la importancia y relevancia del concepto en el trabajo.

\textbf{Época.} En inglés llamada \emph{epoch} es el nombre que se le da al proceso de iterar sobre todas las instancias de entrenamiento, tanto en propagación hacia adelante como hacia atrás. Generalmente un modelo se itera sobre varias épocas a través de los datos; una única vista a cada instancia de datos puede no ser suficiente para que el modelo aprenda las características que darán los mejores resultados.

\textbf{Tipos de errores en clasificación.} Cuando se clasifica datos en clases existen dos tipos de errores posibles. Un falso positivo es un error tipo I, es decir se rechaza una hipótesis nula verdadera. Un error de tipo II por el contrario es un falso negativo, es decir no lograr rechazar una hipótesis nula falsa.

\textbf{Desequilibrio de clases.} Cuando en un problema de clasificación uno de los resultados es mucho más común que el resto se dice que hay un problema de desequilibrio de clases. Esto puede llevar a complicaciones en la generalización del modelo o en la evaluación del desempeño del mismo.

\textbf{\gls{matriz de confusion}.} En problemas de clasificación se tienen distintas medidas de las cuales la más común es la precisión del modelo. Esta medida puede no llegar a ser muy representativa del desempeño del modelo. Una situación en donde esto se podría manifestar es cuando los datos que representan el problema tienen un problema de desequilibrio de clases. Si la tasa de desequilibrio es demasiado alta, es posible predecir únicamente una clase con un modelo y obtener una precisión del modelo bastante elevada a pesar de que el modelo no está prediciendo nada. Debido a esto es importante tener en cuenta más herramientas de retroalimentación. Algunos ejemplos de ellas son:

\begin{itemize}
\item Exhaustividad
\item Métrica de $F_{\beta}$
\item Curvas ROC
\end{itemize}

\begin{table}
\centering
\begin{tabular}{ r | c  r}
 & \multicolumn{2}{ c }{Referencia} \\
Predicción & 0 & 1 \\
\hline
0 & \textbf{1507} & 19 \\
1 & 34 & \textbf{1201} \\
\end{tabular}
\caption{Ejemplo de una matriz de confusión que muestra distintos tipos de errores. La tabla tiene únicamente propósitos ilustrativos. Los números resaltados representan la cantidad de predicciones correctas; las otras dos cifras serán los errores de tipo I y de tipo II.}
\label{table:confmatrix}
\end{table}

% Chapter 3

\chapter{Transferencia de aprendizaje} % Main chapter title

\label{Chapter3} % For referencing the chapter elsewhere, use \ref{Chapter1} 

Un concepto esencial para este trabajo y para toda el área de aprendizaje profundo es la capacidad de transferir aprendizaje adquirido previamente de un problema poco acotado y poder lograr aprovechar este aprendizaje en problemas más acotados permitiendo así que las demandas de recursos como datos y tiempo de ejecución tengan más holgura.

\subsection{Concepto de transferencia}



\subsection{Aplicación en otras áreas}
%Incluir trabajos relacionados
Para tener el contexto acerca de este concepto crucial para este trabajo debemos explorar otras áreas del aprendizaje profundo, en particular el área de visión artificial (computer vision en inglés). Este campo fue el que llevó a la explosión del aprendizaje profundo en el 2012 cuando una red denominada AlexNet dominó la competencia de ImageNet.

\textbf{ImageNet.} Este proyecto es una colección masiva de imágenes que están debidamente etiquetadas con cuadros encerrando los objetos detectados en una imagen. Debido a la cantidad exagerada de categorías y objetos encontrados en las imágenes y también la cantidad en sí de imágenes, se organiza un concurso de software todos los años para determinar cuál modelo es el mejor en visión artificial. Cuando en el 2012 una red de aprendizaje profundo ganó el concurso se generó motivación en explorar este campo más a fondo.

Esta tarea llegó a ser el estándar para poder determinar si un modelo podía \emph{ver} a nivel general. ¿Qué significa esto? La tarea de ver es una tarea muy basta y bastante general. Hay muchas características por considerar lo cual aumenta la complejidad de este problema.

% incluir ilustracion de imagenet aca

En CV existen muchas más tareas más acotadas que ver. Para estas sería muy útil poder utilizar información previamente aprendida con la finalidad no necesitar muchos datos para la tarea más específica. Por su misma naturlaza tendrá menos datos de entrenamiento.

\subsection{Aplicación en NLP}

\subsection{ULMFiT}
% Chapter 4

\chapter{Datos utilizados}

\label{Chapter4} % For referencing the chapter elsewhere, use \ref{Chapter1} 

``On two occasions I have been asked, "Pray, Mr. Babbage, if you put into the machine wrong figures, will the right answers come out?" ... I am not able rightly to apprehend the kind of confusion of ideas that could provoke such a question.''

\hfill ---Charles Babbage, Passages from the Life of a Philosopher

El concepto de \emph{garbage in, garbage out} (basura para adentro, basura para afuera) es muy conocido e indica como un modelo, análisis, etc. es limitado en su capacidad por los datos que se usen para alimentarlo.

En proyectos de aprendizaje artificial se dice que el 80\% del tiempo se deberá usar para la limpieza y tratado de datos. Esta proporción no se mantiene cuando se lidia con modelos de aprendizaje profundo en muchos casos pero el principio es el mismo: no se debe subestimar la importancia de los datos y su integridad.


\section{Pre-procesamiento de datos}

\subsection{Datos tabulares}

\begin{table}
\centering
\begin{tabular}{c c c c | c c}
CORRELATIVO & Edad & Género & País & Sueldo & Moneda \\
\hline
1 & 24 & F & GT & 3500 & GTQ \\
2 & 145 & M & Perú & 100500 & Soles \\
3 & 35 & N/A & .es & 1.5 & \texteuro \\
& & & & & \\
$\vdots$ &  & $\ddots$ & & & $\vdots$ \\
& & & & & \\
2304 & 18 & F & GT & 2.8k & Q \\
\end{tabular}
\label{table:tabulares}
\caption{Ejemplo de datos tabulares que alimentarán a un modelo ya sea de aprendizaje de máquina o de aprendizaje profundo. En la tabla se ilustran algunos errores de datos (campo \emph{edad}) y datos faltantes (en el campo \emph{Género}). También se aprecian discrepancias en el formato de algunos campos (campo de \emph{Moneda} y \emph{Sueldo}) y campos completamente inservibles para un modelo (campo \emph{CORRELATIVO}.}
\end{table}

La recolección de datos para la alimentación de modelos por lo general es un proceso que involucra obtener información de distintas fuentes con distintos formatos y distintos detalles que esperar. El pre-procesamiento de datos se encarga de darle una forma congruente a los datos para que estos puedan representar el problema de forma adecuada. Este proceso puede involucrar eliminar errores de formato, identificar datos atípicos y lidiar con ellos, identificar características valiosas para un modelo, etc. Estos pasos son más aparentes cuando se lidia con datos tabulares, es decir datos que están separados por categorías o columnas y deben ser tratados de esta manera tratando cada una como una variable.

\subsection{Datos textuales}

En NLP y en este trabajo se lidia con datos en forma de texto y no de forma tabular. Esto conlleva un conjunto de retos especiales a considerar. Surgen conceptos nuevos a considerar. En NLP los pasos que generalmente se realizan para limpiar los datos son los siguientes:

\begin{itemize}
\item \textbf{Tokenizar.} Esta técnica de pre-procesamiento consiste en separar las cadenas de palabras --- textos --- en \emph{tokens}. Esto con el propósito de que el modelo sea capaz de digerir las cadenas de palabras por segmentos bien definidos. El delimitador de los tokens en los datos de entrada podrá ser algo simple como un caracter de espacio (' '). Existen también técnicas más elaboradas. Por ejemplo, tokenización basado en la morfología de un lenguaje.
\item \textbf{Limitar vocabulario.} Como consecuencia de tener cantidades grandes de datos para la alimentación del modelo se puede terminar teniendo un vocabulario bastante numeroso. Este vocabulario está compuesto de todos los tokens únicos que se encontraron en los datos de entrada. Una técnica común para evitar utilizar tokens \emph{vistos} muy pocas veces en la data es limitar el número de tokens a utilizar. Se elige esa cantidad $n$ de datos a utilizar ordenando los tokens por su frecuencia y tomando los $n$ más comunes.
\item \textbf{Eliminar palabras con propósito gramático.} En muchas tareas de clasificación en NLP dependen mucho del contenido de los textos. Una técnica común para poder enfatizar el valor semántico de los textos es eliminar tokens cuyo propósito es únicamente gramatical. Se podrían nombrar artículos y preposiciónes como ejemplo.
\item \textbf{Vectorizar.} Una técnica muy valiosa que consiste en representar las cadenas de caracteres en vectores en un espacio de $n$ dimensión. Mientras más grande $n$, mejores las posibilidades de capturar más el significado de cada token o palabra. En modelos del estado del arte en NLP se utiliza una dimensionalidad generalmente de $n = 300$ o $n = 400$. Algunas técnicas de vectorización son capaces de capturar mucha información semántica, al grado de poder inferir información acerca de incluso tokens no antes vistos.
\item \textbf{Zero padding.} Esta es una técnica utilizada para dar a los datos de entrada el mismo largo y permitir que el modelo no sufra por la variabilidad de longitud. Esta técnica es más común cuando se utilizan modelos de ML y no DL.
\item \textbf{Tokens especiales.} Al aplicar técnicas como limitación de vocabulario o zero padding surge la necesidad de usar tokens especiales que le indican al modelo conceptos como un término no antes visto o el final de una secuencia. Estos tokens especiales generalmente son definidos por quien realiza el proceso de tokenización aunque también se pueden incluir previo a esta etapa. Estos tokens especiales generalmente son representados en este estilo \texttt{<unk>} donde \texttt{unk} es generalmente una abreviación del concepto del token especial. 
\end{itemize}

\section{Fuente de los datos}

LEFTy es un proyecto multi-tarea, lo cual significa que las predicciones son de distintos conceptos. Esto generalmente lleva a utilizar datos distintos para cada tarea en específico. Gracias a la ventaja que la transferencia de aprendizaje provee --- capítulo \ref{Chapter3} --- la cantidad de datos necesaria para obtener buenos resultados no era excesiva y esto facilitó la busqueda de datos.

Las fuentes de los datos para cada subtarea fueron los siguientes:

\begin{comment}
incluir fuentes del readme aca
\end{comment}



% Chapter 5

\chapter{LEFTy}

\label{Chapter5} % For referencing the chapter elsewhere, use \ref{Chapter5} 

LEFTy --- por sus siglas en inglés: Language Efficient Text Portray --- es el nombre designado para referirse al trabajo actual. Esta solución propuesta emplea el concepto de \textit{transfer learning} para poder permitir entrenar con gran capacidad tareas que no tienen muchos ejemplos etiquetados. Utiliza una RNN como base del modelo y las características base obtenidas fueron:

\begin{itemize}
\item \textbf{Edad.}
\item \textbf{Género.}
\item \textbf{Región de origen.}
\end{itemize}

\section{Pre-entrenamiento de modelo de lenguaje}

La fase de pre-entrenamiento en el contexto de NLP consiste en entrenar una especie de modelo de lenguaje. En el artículo original de ULMFiT \parencite{howard2018}, se utiliza un modelo estándar en donde se predice el siguiente token basado en una cadena de tokens. BERT \parencite{devlin2018bert} por otro lado utiliza un Masked Language Model (MLM) el cual consiste en predecir el 15\% de los tokens dado todo el contexto que los rodea.

\subsection{Modelo de lenguaje de Wikipedia}

El diseño base para este modelo de lenguaje es una red denominada AWD LSTM \parencite{merityRegOpt}.
la cual es una modificación agresiva al método de regularización llamado \textit{dropout}. En el artículo se sugiere utilizar un concepto denominado \textit{DropConnect} y difiere en \textit{dropout} en que las funciones de activación no son las que toman el valor cero, sino los pesos. También se utilizan los conceptos de usar \textit{dropout} en la capa de vectorización de palabras --- esto no aporta a la regularización pero sí disminuye el tamaño de los vectores representantes de los vectores. Se instancian los distintos tipos de \textit{dropout} con pesos asignados a cada uno. En el artículo se recomiendan usar ciertos pesos base y optimizar un hiperparámetro $w_f$ únicamente el cual le da escala a los pesos recomendados y definidos por ellos.

En el capítulo \ref{Chapter4} se explica el pre-procesamiento que se le da a los datos de \textit{Wikipedia}. Se detallará ese proceso a continuación.

Para realizar este procedimiento se utilizaron los recursos de \emph{Google Colaboratory} (Colab), los cuales ofrecen un ambiente de \textit{Notebooks} de IPython y la habilidad de ejecutar comandos de \*nix.

\begin{figure}
\includegraphics[scale=0.3]{Figures/wikidump.pdf}
\caption{Estructura de datos resultante al extraer un archivo de \textit{Wikipedia}.}
\end{figure}
\label{fig:wikidump}


\textbf{Obtención de datos.} El corpus de Wikipedia fue obtenido del sitio oficial (\url{https://dumps.wikimedia.org/eswiki/}). El corpus obtenido fue de noviembre 2018. Estos archivos tienen una estructura específica y la forma recomendada de extraer sus contenidos es utilizando \emph{WikiExtractor} (\url{https://github.com/attardi/wikiextractor}). Esta herramienta permite realizar una extracción que filtra por un parámetro de mínimo de longitud del artículo. Se utilizó este parámetro para filtrar todos los artículos con menos de 1000 palabras.

Una vez finaliza la extracción del archivo --- la cual demora una cantidad no despreciable de horas --- se procede a leer y filtrar los documentos. En el caso de este trabajo se filtraron todos los documentos después de haber acumulado 100 millones de tokens en lo seleccionado. Se conservaron 10 millones de tokens adicionales para la validación de resultados.

\textbf{Tokenización.} La herramienta utilizada para este proceso fue spaCy (\url{https://spacy.io/}). Esta herramienta tiene soporte para más de 34 idiomas, entre los cuales está incluído el español.

\subsection{Recursos utilizados para entrenamiento}

\textbf{Costo monetario.} Para esta fase de entrenamiento se recurre a los servicios de \textit{Google Cloud} (\href{https://cloud.google.com/free}{https://cloud.google.com}) los cuales son ofrecidos con un beneficio de 300 USD para utilizar durante el primer año. No es necesario ser estudiante o profesor para gozar de este beneficio. Teniendo estos recursos disponibles se optó por utilizar una instance de cómputo \emph{n1-highmem-4} la cual cuenta con	el siguiente hardware para el entrenamiento:
\begin{itemize}
\item 4 vCPUs
\item 26 GB de memoria (RAM)
\item nVIDIA V100 GPU
\end{itemize}

Lo primordial cuando se trata de entrenamiento de RNNs es la capacidad de cómputo de la GPU. La \textbf{V} en el modelo V100 indica que es de la última generación a la fecha de esta tesis y provee una ventaja significativa comparada con una K80 o P100.

\textbf{Costo de recursos.} El costo total resultante después de entrenar un modelo inicial y funcional llegó a \$ 60.20 USD. Esto fue cubierto por los créditos iniciales ofrecidos por Google. También se debe considerar que este paso se debe realizar \emph{una sola vez} para cada lenguaje. En caso de querer utilizar el modelo entrenado en este trabajo, se podrá hacerlo y se podrá aplicar a otros problemas de clasificación. El modelo se encuentra bajo dominio público en este dominio:

\textbf{Costo en tiempo.} Para entrenar los modelos de lenguaje con una estructura AWD LSTM, una época demoraba alrededor de una hora. Después de 4 épocas ya se aproximaban los resultados al estado del arte y puede decidirse si continuar o no.








%% Chapter 6

\chapter{Conclusiones}

\label{Chapter6} % For referencing the chapter elsewhere, use \ref{Chapter1} 

\section{Conclusiones}

Conclusiones del trabajo

\section{Trabajo futuro}

Usar transformadores. Ya sea el nuevo Transformer-XL o BERT.

Refinar proceso de limpieza de datos. En particular detectar retuits, respuestas (ignorar mentions), etc.

Concatenación de textos, no para avanzar la tarea específica sino el resultado para la métrica específica de PAN17.

Considerar un modelo multitask, es decir que la misma red tenga output de todas las características definidas(?).


%----------------------------------------------------------------------------------------
%	THESIS CONTENT - APPENDICES
%----------------------------------------------------------------------------------------

%\appendix % Cue to tell LaTeX that the following "chapters" are Appendices

% Include the appendices of the thesis as separate files from the Appendices folder
% Uncomment the lines as you write the Appendices

%% Appendix A

\printglossary[type=main, title=Glosario]


%\include{Appendices/AppendixB}
%\include{Appendices/AppendixC}

%----------------------------------------------------------------------------------------
%	BIBLIOGRAPHY
%----------------------------------------------------------------------------------------

\nocite{*}
\printbibliography[heading=bibintoc]

%----------------------------------------------------------------------------------------

\end{document}  
